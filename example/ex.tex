% To compile this, do the following:
%
%   pdflatex ex.tex
%   bibtex ex
%   pdflatex ex.tex


\documentclass[conference]{IEEEtran} % IEEE conference format

\usepackage{cite}       % Citation management
\usepackage{graphicx}   % For figures
\usepackage{amsmath,amssymb} % For math symbols
\usepackage{hyperref}   % For hyperlinks
\usepackage{lipsum}     % For Lorem Ipsum text

\begin{document}

\title{Bubbleditty: A Low-Energy FTL Spacetime Warp Engine Powered by a Lead-Acid Battery}

\author{
    \IEEEauthorblockN{Ricard Oppenstein}
    \IEEEauthorblockA{
        Department of Warp Field Engineering \\
        Institute for Temporal Dynamics \\
        Email: ricard.oppenstein@itd.edu
    }
    \and
    \IEEEauthorblockN{Maria Anning Meitner}
    \IEEEauthorblockA{
        Division of Exotic Propulsion \\
        University of Quantum Paleontology \\
        Email: maria.meitner@uqp.edu
    }
}
\maketitle

\begin{abstract}
    \lipsum[1] % Placeholder text for abstract
\end{abstract}

\section{Introduction}
Example introduction...  Faster-than-light (FTL) travel has been a cornerstone of science fiction for decades. However, recent studies\cite{Pu2013_WiSee} suggest that spacetime distortions could allow for such travel within the framework of general relativity. We propose a novel, low-energy approach utilizing a conventional lead-acid battery.

\section{Experimental Setup}
We use a 12V lead-acid battery to power a miniature warp coil. The system consists of:
\begin{itemize}
    \item A superconducting ring of exotic matter.
    \item A flux capacitor to stabilize spacetime perturbations.
    \item A small-scale prototype tested within a controlled vacuum environment.
\end{itemize}

\section{Results}
Preliminary simulations indicate that a properly tuned lead-acid battery could sustain a localized warp effect for 42 milliseconds before thermal dissipation.

\section{Conclusion}
This paper outlines a theoretical framework for an FTL engine powered by everyday lead-acid batteries. Further research is required to confirm practical feasibility.

\section*{Acknowledgments}
The authors would like to thank the Institute for Exotic Engineering for funding this study.

\bibliographystyle{IEEEtran}
\bibliography{../papers/fling.bib} % Bibliography file name

\end{document}



